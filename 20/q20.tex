\documentclass[]{article}
\usepackage{amsfonts, amssymb, amsmath}
\usepackage{float}

\title{Question 10.13.3.20}
\author{Shaikh Rusna Aiyubbhai \\ EE22BTECH11048}
\date{}
\begin{document}
\maketitle
\providecommand{\pr}[1]{\ensuremath{\Pr\left(#1\right)}}
\providecommand{\prt}[2]{\ensuremath{p_{#1}^{\left(#2\right)} }}        % own macro for this question
\providecommand{\qfunc}[1]{\ensuremath{Q\left(#1\right)}}
\providecommand{\sbrak}[1]{\ensuremath{{}\left[#1\right]}}
\providecommand{\lsbrak}[1]{\ensuremath{{}\left[#1\right.}}
\providecommand{\rsbrak}[1]{\ensuremath{{}\left.#1\right]}}
\providecommand{\brak}[1]{\ensuremath{\left(#1\right)}}
\providecommand{\lbrak}[1]{\ensuremath{\left(#1\right.}}
\providecommand{\rbrak}[1]{\ensuremath{\left.#1\right)}}
\providecommand{\cbrak}[1]{\ensuremath{\left\{#1\right\}}}
\providecommand{\lcbrak}[1]{\ensuremath{\left\{#1\right.}}
\providecommand{\rcbrak}[1]{\ensuremath{\left.#1\right\}}}
\newcommand{\sgn}{\mathop{\mathrm{sgn}}}
\providecommand{\abs}[1]{\left\vert#1\right\vert}
\providecommand{\res}[1]{\Res\displaylimits_{#1}} 
\providecommand{\norm}[1]{\left\lVert#1\right\rVert}
%\providecommand{\norm}[1]{\lVert#1\rVert}
\providecommand{\mtx}[1]{\mathbf{#1}}
\providecommand{\mean}[1]{E\left[ #1 \right]}
\providecommand{\cond}[2]{#1\middle|#2}
\providecommand{\fourier}{\overset{\mathcal{F}}{ \rightleftharpoons}}
\newenvironment{amatrix}[1]{%
  \left(\begin{array}{@{}*{#1}{c}|c@{}}
}{%
  \end{array}\right)
}
%\providecommand{\hilbert}{\overset{\mathcal{H}}{ \rightleftharpoons}}
%\providecommand{\system}{\overset{\mathcal{H}}{ \longleftrightarrow}}
	%\newcommand{\solution}[2]{\textbf{Solution:}{#1}}
\newcommand{\solution}{\noindent \textbf{Solution: }}
\newcommand{\cosec}{\,\text{cosec}\,}
\providecommand{\dec}[2]{\ensuremath{\overset{#1}{\underset{#2}{\gtrless}}}}
\newcommand{\myvec}[1]{\ensuremath{\begin{pmatrix}#1\end{pmatrix}}}
\newcommand{\mydet}[1]{\ensuremath{\begin{vmatrix}#1\end{vmatrix}}}
\newcommand{\myaugvec}[2]{\ensuremath{\begin{amatrix}{#1}#2\end{amatrix}}}
\providecommand{\rank}{\text{rank}}
\providecommand{\pr}[1]{\ensuremath{\Pr\left(#1\right)}}
\providecommand{\qfunc}[1]{\ensuremath{Q\left(#1\right)}}
	\newcommand*{\permcomb}[4][0mu]{{{}^{#3}\mkern#1#2_{#4}}}
\newcommand*{\perm}[1][-3mu]{\permcomb[#1]{P}}
\newcommand*{\comb}[1][-1mu]{\permcomb[#1]{C}}
\providecommand{\qfunc}[1]{\ensuremath{Q\left(#1\right)}}
\providecommand{\gauss}[2]{\mathcal{N}\ensuremath{\left(#1,#2\right)}}
\providecommand{\diff}[2]{\ensuremath{\frac{d{#1}}{d{#2}}}}
\providecommand{\myceil}[1]{\left \lceil #1 \right \rceil }
\newcommand\figref{Fig.~\ref}
\newcommand\tabref{Table~\ref}
\newcommand{\sinc}{\,\text{sinc}\,}
\newcommand{\rect}{\,\text{rect}\,}
%%
%	%\newcommand{\solution}[2]{\textbf{Solution:}{#1}}
%\newcommand{\solution}{\noindent \textbf{Solution: }}
%\newcommand{\cosec}{\,\text{cosec}\,}
%\numberwithin{equation}{section}
%\numberwithin{equation}{subsection}
%\numberwithin{problem}{section}
%\numberwithin{definition}{section}
%\makeatletter
%\@addtoreset{figure}{problem}
%\makeatother

%\let\StandardTheFigure\thefigure
\let\vec\mathbf

20. Two dice are thrown simultaneously. What is the probability that the sum of the numbers appearing on the dice is (\romannumeral1) 7? (\romannumeral2) a prime number? (\romannumeral3) 1?

\solution Let X and Y represent number appearing on two dice. Let Z be the sum of the numbers appearing on two dice.
$$Z=X+Y$$

\begin{table}[H]
\centering
\begin{tabular}{|c|c|}
\hline
random variables & description \\
\hline
X & number appearing on first dice \\
\hline
Y & number appearing on second dice \\
\hline
Z & sum of numbers appearing on both dice \\
\hline
\end{tabular}
\label{tab:ncert_exemplar/10/13/3/20}
\end{table}

 We know,
\begin{align}
\Pr(Z=n) = 
\begin{cases}
0 &  n \leq{1} \\
\frac{n-1}{36} &  2 \leq{x} \leq{7} \\
\frac{13-n}{36} &  7 < n \leq{12} \\
0 &  n>12
\label{eq:ncert_exemplar/10/13/3/20}
\end{cases}
\end{align}

Then,

(\romannumeral1) The sum of numbers appearing on the dice is 7. Then from \eqref{eq:ncert_exemplar/10/13/3/20},
\begin{align}
\Pr(Z=7) &= \frac{7-1}{36} \\
&= \frac{1}{6}
\end{align}

(\romannumeral2) The sum of numbers appearing on dice is a prime number. From \eqref{eq:ncert_exemplar/10/13/3/20},
\begin{align}
\Pr(Z=\text{prime number}) &= Pr(Z=2) + \Pr(Z=3) + \Pr(Z=5) + \Pr(Z=7) + \Pr(Z=11) \\
&= \frac{1}{36} + \frac{2}{36} + \frac{4}{36} + \frac{6}{36} + \frac{2}{36} \\
&= \frac{15}{36} \\
&= \frac{5}{12}
\end{align}

(\romannumeral3) From \eqref{eq:ncert_exemplar/10/13/3/20}, the probability of the sum of numbers appearing on the dice is 1 is, 
\begin{align}
\Pr(Z=1) &= 0
\end{align}
\end{document}
