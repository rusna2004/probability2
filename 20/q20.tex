\documentclass[]{article}
\usepackage{amsfonts, amssymb, amsmath}
\usepackage{float}
\begin{document}
\providecommand{\pr}[1]{\ensuremath{\Pr\left(#1\right)}}
\providecommand{\prt}[2]{\ensuremath{p_{#1}^{\left(#2\right)} }}        % own macro for this question
\providecommand{\qfunc}[1]{\ensuremath{Q\left(#1\right)}}
\providecommand{\sbrak}[1]{\ensuremath{{}\left[#1\right]}}
\providecommand{\lsbrak}[1]{\ensuremath{{}\left[#1\right.}}
\providecommand{\rsbrak}[1]{\ensuremath{{}\left.#1\right]}}
\providecommand{\brak}[1]{\ensuremath{\left(#1\right)}}
\providecommand{\lbrak}[1]{\ensuremath{\left(#1\right.}}
\providecommand{\rbrak}[1]{\ensuremath{\left.#1\right)}}
\providecommand{\cbrak}[1]{\ensuremath{\left\{#1\right\}}}
\providecommand{\lcbrak}[1]{\ensuremath{\left\{#1\right.}}
\providecommand{\rcbrak}[1]{\ensuremath{\left.#1\right\}}}
\newcommand{\sgn}{\mathop{\mathrm{sgn}}}
\providecommand{\abs}[1]{\left\vert#1\right\vert}
\providecommand{\res}[1]{\Res\displaylimits_{#1}} 
\providecommand{\norm}[1]{\left\lVert#1\right\rVert}
%\providecommand{\norm}[1]{\lVert#1\rVert}
\providecommand{\mtx}[1]{\mathbf{#1}}
\providecommand{\mean}[1]{E\left[ #1 \right]}
\providecommand{\cond}[2]{#1\middle|#2}
\providecommand{\fourier}{\overset{\mathcal{F}}{ \rightleftharpoons}}
\newenvironment{amatrix}[1]{%
  \left(\begin{array}{@{}*{#1}{c}|c@{}}
}{%
  \end{array}\right)
}
%\providecommand{\hilbert}{\overset{\mathcal{H}}{ \rightleftharpoons}}
%\providecommand{\system}{\overset{\mathcal{H}}{ \longleftrightarrow}}
	%\newcommand{\solution}[2]{\textbf{Solution:}{#1}}
\newcommand{\solution}{\noindent \textbf{Solution: }}
\newcommand{\cosec}{\,\text{cosec}\,}
\providecommand{\dec}[2]{\ensuremath{\overset{#1}{\underset{#2}{\gtrless}}}}
\newcommand{\myvec}[1]{\ensuremath{\begin{pmatrix}#1\end{pmatrix}}}
\newcommand{\mydet}[1]{\ensuremath{\begin{vmatrix}#1\end{vmatrix}}}
\newcommand{\myaugvec}[2]{\ensuremath{\begin{amatrix}{#1}#2\end{amatrix}}}
\providecommand{\rank}{\text{rank}}
\providecommand{\pr}[1]{\ensuremath{\Pr\left(#1\right)}}
\providecommand{\qfunc}[1]{\ensuremath{Q\left(#1\right)}}
	\newcommand*{\permcomb}[4][0mu]{{{}^{#3}\mkern#1#2_{#4}}}
\newcommand*{\perm}[1][-3mu]{\permcomb[#1]{P}}
\newcommand*{\comb}[1][-1mu]{\permcomb[#1]{C}}
\providecommand{\qfunc}[1]{\ensuremath{Q\left(#1\right)}}
\providecommand{\gauss}[2]{\mathcal{N}\ensuremath{\left(#1,#2\right)}}
\providecommand{\diff}[2]{\ensuremath{\frac{d{#1}}{d{#2}}}}
\providecommand{\myceil}[1]{\left \lceil #1 \right \rceil }
\newcommand\figref{Fig.~\ref}
\newcommand\tabref{Table~\ref}
\newcommand{\sinc}{\,\text{sinc}\,}
\newcommand{\rect}{\,\text{rect}\,}
%%
%	%\newcommand{\solution}[2]{\textbf{Solution:}{#1}}
%\newcommand{\solution}{\noindent \textbf{Solution: }}
%\newcommand{\cosec}{\,\text{cosec}\,}
%\numberwithin{equation}{section}
%\numberwithin{equation}{subsection}
%\numberwithin{problem}{section}
%\numberwithin{definition}{section}
%\makeatletter
%\@addtoreset{figure}{problem}
%\makeatother

%\let\StandardTheFigure\thefigure
\let\vec\mathbf

20. Two dice are thrown simultaneously. What is the probability that the sum of the numbers appearing on the dice is (\romannumeral1) 7? (\romannumeral2) a prime number? (\romannumeral3) 1?

\solution When one die shows `1', the other die could show any one of the numbers 1,2,3,4,5,6. The same is true when the first die shows `2', `3', `4', `5' or `6'. The possible outcomes of the experiment are listed below; the first number in each ordered pair is the number appearing on the first die and the second number is that on the other die.

\begin{table}[H]
\centering
\begin{tabular}{c|c c c c c c}
&1&2&3&4&5&6\\
\hline
1&(1,1)&(1,2)&(1,3)&(1,4)&(1,5)&(1,6)\\
2&(2,1)&(2,2)&(2,3)&(2,4)&(2,5)&(2,6)\\
3&(3,1)&(3,2)&(3,3)&(3,4)&(3,5)&(3,6)\\
4&(4,1)&(4,2)&(4,3)&(4,4)&(4,5)&(4,6)\\
5&(5,1)&(5,2)&(5,3)&(5,4)&(5,5)&(5,6)\\
6&(6,1)&(6,2)&(6,3)&(6,4)&(6,5)&(6,6)\\
\end{tabular}
\label{tab:outcomes}
\end{table}

So, the number of possible outcomes $=6 \times 6=36$

(\romannumeral1) The outcomes favourable to the event `the sum of the two numbers is 7' denoted by E are: (1,6), (2,5), (3,4), (4,3), (5,2), (6,1)  (see figure above) \\
i.e., the number of favourable outcomes to E = 5.
Hence,
\begin{align}
P(E) &= \frac{6}{36}\\
&= \frac{1}{6}
\end{align}

(\romannumeral2) As you can see from figure above, the outcomes of the event F, `the sum of the two numbers is a prime number' are: (1,1), (1,2), (1,4), (1,6), (2,1), (2,3), (2,5), (3,2), (3,4), (4,1), (4,3), (5,2), (5,6), (6,1), (6,5) \\
i.e., the number of favourable outcomes to F = 15 \\
So,
\begin{align}
P(F) &= \frac{15}{36}\\
&= \frac{5}{12}
\end{align}

(\romannumeral3) As you can see from figure above, there is no outcome favourable to the event G, `the sum of two numbers is 1'.\\
So,
\begin{align}
P(G) &= \frac{0}{36} \\
&= 0 
\end{align}
\end{document}
